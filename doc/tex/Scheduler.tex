\section{Scheduler}
\label{scheduler}

The MTL runtime maps N actors to M threads on the local machine. Applications
build with MTL scale by decomposing tasks into many independent steps that are
spawned as actors. In this way, sequential computations performed by individual
actors are small compared to the total runtime of the application, and the
attainable speedup on multi-core hardware is maximized in agreement with
Amdahl's law.

Decomposing tasks implies that actors are often short-lived. Assigning a
dedicated thread to each actor would not scale well. Instead, MTL includes a
scheduler that dynamically assigns actors to a pre-dimensioned set of worker
threads. Actors are modeled as lightweight state machines. Whenever a
\emph{waiting} actor receives a message, it changes its state to \emph{ready}
and is scheduled for execution. MTL cannot interrupt running actors because it
is implemented in user space. Consequently, actors that use blocking system
calls such as I/O functions can suspend threads and create an imbalance or lead
to starvation. Such ``uncooperative'' actors can be explicitly detached by the
programmer by using the \lstinline^detached^ spawn option, e.g.,
\lstinline^system.spawn<detached>(my_actor_fun)^.

The performance of actor-based applications depends on the scheduling algorithm
in use and its configuration. Different application scenarios require different
trade-offs. For example, interactive applications such as shells or GUIs want
to stay responsive to user input at all times, while batch processing
applications demand only to perform a given task in the shortest possible time.

Aside from managing actors, the scheduler bridges actor and non-actor code. For
this reason, the scheduler distinguishes between external and internal events.
An external event occurs whenever an actor is spawned from a non-actor context
or an actor receives a message from a thread that is not under the control of
the scheduler. Internal events are send and spawn operations from scheduled
actors.

\subsection{Policies}
\label{scheduler-policy}

The scheduler consists of a single coordinator and a set of workers. The
coordinator is needed by the public API to bridge actor and non-actor contexts,
but is not necessarily an active software entity.

The scheduler of MTL is fully customizable by using a policy-based design. The
following class shows a \emph{concept} class that lists all required member
types and member functions. A policy provides the two data structures
\lstinline^coordinator_data^ and \lstinline^worker_data^ that add additional
data members to the coordinator and its workers respectively, e.g., work
queues. This grants developers full control over the state of the scheduler.

\begin{lstlisting}
struct scheduler_policy {
  struct coordinator_data;
  struct worker_data;
  void central_enqueue(Coordinator* self, resumable* job);
  void external_enqueue(Worker* self, resumable* job);
  void internal_enqueue(Worker* self, resumable* job);
  void resume_job_later(Worker* self, resumable* job);
  resumable* dequeue(Worker* self);
  void before_resume(Worker* self, resumable* job);
  void after_resume(Worker* self, resumable* job);
  void after_completion(Worker* self, resumable* job);
};
\end{lstlisting}

Whenever a new work item is scheduled---usually by sending a message to an idle
actor---, one of the functions \lstinline^central_enqueue^,
\lstinline^external_enqueue^, and \lstinline^internal_enqueue^ is called. The
first function is called whenever non-actor code interacts with the actor
system. For example when spawning an actor from \lstinline^main^. Its first
argument is a pointer to the coordinator singleton and the second argument is
the new work item---usually an actor that became ready. The function
\lstinline^external_enqueue^ is never called directly by MTL. It models the
transfer of a task to a worker by the coordinator or another worker. Its first
argument is the worker receiving the new task referenced in the second
argument. The third function, \lstinline^internal_enqueue^, is called whenever
an actor interacts with other actors in the system. Its first argument is the
current worker and the second argument is the new work item.

Actors reaching the maximum number of messages per run are re-scheduled with
\lstinline^resume_job_later^ and workers acquire new work by calling
\lstinline^dequeue^. The two functions \lstinline^before_resume^ and
\lstinline^after_resume^ allow programmers to measure individual actor runtime,
while \lstinline^after_completion^ allows to execute custom code whenever a
work item has finished execution by changing its state to \emph{done}, but
before it is destroyed. In this way, the last three functions enable developers
to gain fine-grained insight into the scheduling order and individual execution
times.

\subsection{Work Stealing}
\label{work-stealing}

The default policy in MTL is work stealing. The key idea of this algorithm is
to remove the bottleneck of a single, global work queue.  The original
algorithm was developed for fully strict computations by Blumofe et al in 1994.
It schedules any number of tasks to \lstinline^T^ workers, where \lstinline^T^
is the number of processors available.

\singlefig{stealing}{Stealing of work items}{fig-stealing}

Each worker dequeues work items from an individual queue until it is drained.
Once this happens, the worker becomes a \emph{thief}. It picks one of the other
workers---usually at random---as a \emph{victim} and tries to \emph{steal} a
work item. As a consequence, tasks (actors) are bound to workers by default and
only migrate between threads as a result of stealing. This strategy minimizes
communication between threads and maximizes cache locality. Work stealing has
become the algorithm of choice for many frameworks. For example, Java's
Fork-Join (which is used by Akka), Intel's Threading Building Blocks, several
OpenMP implementations, etc.

MTL uses a double-ended queue for its workers, which is synchronized with two
spinlocks. One downside of a decentralized algorithm such as work stealing is,
that idle states are hard to detect. Did only one worker run out of work items
or all? Since each worker has only local knowledge, it cannot decide when it
could safely suspend itself. Likewise, workers cannot resume if new job items
arrived at one or more workers. For this reason, MTL uses three polling
intervals. Once a worker runs out of work items, it tries to steal items from
others. First, it uses the \emph{aggressive} polling interval. It falls back to
a \emph{moderate} interval after a predefined number of trials. After another
predefined number of trials, it will finally use a \emph{relaxed} interval.

Per default, the \emph{aggressive} strategy performs 100 steal attempts with no
sleep interval in between. The \emph{moderate} strategy tries to steal 500
times with 50 microseconds sleep between two steal attempts. Finally, the
\emph{relaxed} strategy runs indefinitely but sleeps for 10 milliseconds
between two attempts. These defaults can be overridden via system config at
startup~\see{system-config}.

\subsection{Work Sharing}
\label{work-sharing}

Work sharing is an alternative scheduler policy in MTL that uses a single,
global work queue. This policy uses a mutex and a condition variable on the
central queue. Thus, the policy supports only limited concurrency but does not
need to poll. Using this policy can be a good fit for low-end devices where
power consumption is an important metric.

% TODO: profiling section
